\partie{Mesure et intégration.}
Pour commencer, il faut faire un résumé des résultats et théorémes importants en ce qui concerne la théorie de
la mesure l'intégration par rapport à une mesure. En effet, une probabilité est une mesure particuliére (de masse totale
égale à $1$) et une espérance est une intégrale par rapport à cette probabilité. De plus, nombre de notions de probabilités,
comme une espérance conditionnelle ou une martingale ne peuvent pas s'énoncer hors de ce cadre. Cependant, comme nous avons pu
le dire, nous ne voulons (et ne pouvons) pas réaliser un cours d'intégration mais plutét rappeler des résultats et des définition 
utiles pour la suite.\par
Dans cette partie $X$ désigne toujours un ensemble non vide et $\parties{X}$ l'ensemble de ses parties.
\spartie{Intégrale par rapport é une mesure positive.}
\definition{$\sigma-$algébre.} Un sous ensemble $\mathcal{M}$ de $\parties{X}$ est une tribu ou $\sigma-$algébre
lorsque~:\par
\begin{itemize}
\item{} $\emptyset \in \mathcal{M}$
\item{} $A\in \mmm \implique A^c\in \mmm$
\item{} $\displaystyle{A_n \in \mmm \implique \bigcup_{n\in \nn}A_n\in \mmm}$
\end{itemize}
Des tribus existent toujours puisque $\parties{X}$ en définit une et qu'une intersection de tribus est encore une
tribu.\par
\rmq Cette propriété permet de définir la plus petite tribu contenant une partie $A$ donnée de $\parties{X}$, notée
$\sigma (A)$. Ainsi on parlera de tribu borélienne en désignant la tribu engendrée par les ouverts d'un espace
topologique comme ${\rr}^d$.
\definition{Mesure positive.} C'est une application $\mu$ de $\mmm$ dans $\rbar_+$ qui vérifie deux propriétés~:\par
\begin{itemize}
\item{} $\mu(\emptyset) = 0$
\item{} Si $A_n\in \mmm$, tous disjoints alors $\displaystyle{\mu \bigl(\bigcup_{n\in \nn}A_n\bigr) = 
                                                    \serie{n=0}{+\infty}{\mu(A_n)}}$
\end{itemize}
La premiére propriété n'est pas toujours donnée mais en fait il n'y a que deux valeurs possibles pour $\mu(\emptyset)$, 
$0$ et $+\infty$. La deuxiéme n'ayant aucun intérét, on impose la premiére possibilité.\par
Les mesures positives possédent certaines propriétés importantes comme~:\par
\begin{itemize}
\item{} Si $A_n$ est une suite décroissante d'ensembles mesurables de mesure finie et $A = \cap A_n$ alors~:\par
$$\mu(A) = \limndecroiss \mu(A_n)$$
\item{} Si $A_n$ est une suite croissante d'ensembles mesurables et $A = \cup A_n$ alors~:\par
$$\mu(A) = \limncroiss \mu(A_n)$$
\end{itemize}
On peut donner des exemples de mesures, comme la mesure de dénombrement définie par $\mu(\lbrace{a\rbrace}) = 1$, ou la mesure de 
Lebesgue définie sur la tribu borélienne de $\rr$ par $\mu(\segment{a}{b}) = b-a$.\par
Les deux exemples précédents suggérent qu'il suffit d'imposer les valeurs d'une mesure sur une certaine classe
de sous ensembles pour définir une mesure. Afin de formaliser tout cela, on introduit la notion de classe monotone.
\definition{Classe monotone.} Un sous ensemble $\mmm$ de $\parties{X}$ est une classe monotone lorsque~:
\begin{itemize}
\item{} $X\in \mmm$
\item{} Si $A,B\in \mmm$ et $A\subset B$ alors $B\cap A^c\in \mmm$
\item{} Si $A_n\in \mmm$ et $A_n\subset A_{n+1}$ alors $\bigcup\limits_{n\in \nn} A_n \in \mmm$
\end{itemize}
On vérifie que toute intersection de classe monotone est une classe monotone et on peut donc définir $\mmm(\ccc)$ 
comme la plus petite classe mononotone contenant $\ccc$. On arrive au premier théoréme important~:\par
\theoreme{Lemme de classe monotone.} Si $\ccc$ est stable par intersections finies alors $\mmm(\ccc) = \sigma(\ccc)$.\par
Pour démontrer ce théoréme, il suffit de montrer que $\mmm(\ccc)$ est une tribu et méme seulement qu'elle est stable par 
intersections finies. et donc seulement pour l'intersection. On choisit donc $A \in \ccc$ et on pose~:
$\mmm_1 = \lbrace B\in \mmm(\ccc) : A\cap B\in \mmm(\ccc)\rbrace$ On vérifie que c'est une classe monotone et donc que c'est 
$\mmm(\ccc)$. En reprenant cet argument avec $A\in \mmm(\ccc)$, on conclut.\par
On en déduit un corollaire important, qui justifiera l'unicité de la mesure de Lebesgue ou de la mesure produit.
\theoreme{} Soient $\mu,\nu$ deux mesures sur $(X,\aaa)$. Supposons qu'il existe une classe $\ccc\subset \aaa$ stable par
intersections finies, telle que $\sigma(\ccc) = \aaa$ et $\mu(A) = \nu(A)$ sur $\ccc$. Alors
\begin{itemize}
\item{} Si $\mu(X) = \nu(X)< +\infty$ alors $\mu = \nu$.
\item{} S'il existe une suite croissante $X_n$ de parties de $X$ incluses dans $\ccc$ telles que $\mu(X_n) = \nu(X_n)<\infty$ alors
$\mu = \nu$
\end{itemize}
\dem On pose $\mmm_1 = \lbrace A\in \aaa : \mu(A) = \nu(A)\rbrace$ et on vérifie, dans le cas o\`u $\mu,\nu$ sont finies, que c'est
une classe monotone qui contient $\ccc$. Pour le deuxiéme cas, on pose $\mu_n = \mu(\cdot\cap X_n)$ et de méme pour $\nu$ et on 
applique la premiére partie é ces deux mesures. Pour finir on utilise la propriété de limite croissante des mesures puisque $A = 
\bigcup_n (A\cap X_n)$.\par
On en déduit qu'il existe au plus une mesure positive $\lambda$ sur $(\rr, \bbb(\rr))$ telle que $\lambda(\intervalle{a}{b}) = b-a$.\par
\definition{Fonction mesurable.} Une fonction mesurable est telle que l'image réciproque par $f$ de tout ensemble
mesurable est mesurable.\par
Si on se place sur ${\rr}^{d}$ et qu'on prend la tribu borélienne, une fonction mesurable est telle que l'image
réciproque de tout ouvert, de tout segment ou de tout $\intervalle{-\infty}{a}$ ou $a$ est rationnel soit un 
ensemble mesurable. Ainsi une application continue est mesurable au sens de Borel, ce qui justifie en partie 
la définition. On va maintenant construire l'intégrale par rapport é une mesure d'une fonction mesurable.
\rmq La somme, le produit, la composition de deux fonctions mesurables est mesurable. La borne sup, inf, la limite
simple, la limite sup la limite inf d'une famille dénombrable de fonctions mesurables sont mesurables. L'ensemble
de convergence d'une suite de fonctions est mesurable. (C'est l'image réciproque de la diagonale de ${\rr}^2$ par 
l'application mesurable $\limsup f_n - \liminf f_n$).
\definition{Intégrale d'une fonction étagée.} Une fonction étagée de $X$ dans $\rr$ est une fonction $s$ dont 
l'image est finie. On note $\lbrace{a_i\rbrace}_i$ son image et $A_i = f^{-1}(\lbrace{a_i\rbrace})$. On définit 
alors~:\par
$$\integrale{X}{}{s\dd{\mu}} = \serie{i}{}{a_i\mu(A_i)}$$\par
\rmq Cette définition correspond é la construction de l'intégrale de Riemann mais est plus générale puisque qu'elle
permet d'ores-et-déjé d'intégrer la fonction caractéristique de $\qq\cap\segment{0}{1}$.
\definition{Intégrale d'une fonction positive.} Si $f$ est positive on note $\eee(f)$ l'ensemble des fonctions
en escalier in\-fé\-rieu\-res é $f$ et on pose~:\par
$$\integrale{X}{}{f\dd{\mu}} = \sup_{s\in \eee(f)}\integrale{X}{}{s\dd{\mu}}$$\par
Pour intégrer sur $E\subset X$ il suffit d'appliquer la définition ci-dessus é$f 1_E$.
On conviendra que sur un ensemble de mesure nulle (ou négligeable) l'intégrale de toute fonction est nulle.\par
Pour une fonction quelconque réelle on pose $u^+$ sa partie positive et $u^-$ sa partie négative de sorte que 
$u = u^+ - u^-$.
\definition{Intégrale d'une fonction é valeurs dans $\cc$.}
Une telle fonction $u$ se décompose en $u_1+iu_2$. Ensuite on décompose $u_i = {u_i}^+-{u_i}^-$ et, lorsque 
$\int \lvert f\rvert\dd{\mu}<+\infty$ on définit~:\par
$$\integrale{X}{}{u\dd{\mu}}=\integrale{X}{}{{u_1}^+\dd{\mu}}-\integrale{X}{}{{u_1}^-\dd{\mu}}+
  i\integrale{X}{}{{u_2}^+\dd{\mu}}-i\integrale{X}{}{{u_2}^-\dd{\mu}}$$\par
On vérifie alors que l'ensemble des fonctions intégrables $\lll^1(\mu)$ est un $\cc$-espace vectoriel et que $f\to
\int_X f\dd{\mu}$ y définit une forme linéaire. De plus on a l'utile propriété~:\par
$$\left\lvert\integrale{}{}{f\dd{\mu}}\right\rvert\leq \integrale{}{}{\lvert f\rvert\dd{\mu}}$$\par
\spartie{Théorémes de convergence.}
On suppose que $X, \mu$ est un espace mesuré tout au long de cette sous-partie.
\theoreme{Convergence monotone} Soit $(f_n)_n$ une suite de fonctions mesurables sur $X$ é valeurs dans 
$\segment{0}{\infty}$ telle que $\limncroiss f_n = f$. Alors $f$ est mesurable et\par
$$\limn \integrale{X}{}{f_n\dd{\mu}} = \integrale{X}{}{f\dd{\mu}}$$
\dem Que $f$ soit mesurable provient du fait que c'est le sup d'une famille de fonctions mesurables. Comme la suite
des intégrales de gauche converge dans $\rbar$, on peut noter $\alpha$ sa limite et un passage é la limite donne 
$\alpha\leq\int_X f\dd{\mu}$. Reste é établir l'inégalité inverse. On ne peut évidemment pas le faire directement.
Soit alors $s$ une fonction étagée plus petite que $f$ et $c$ un réel de $\intervalle{0}{1}$. Posons 
$E_n = \lbrace x\in X\lvert f_n(x)\geq c s(x)\rbrace$. Tous ces ensembles sont mesurables, $E_n\subset E_{n+1}$ et
$X = \cup E_n$. Alors~:\par
$\int_X f_n\dd{\mu} \geq \int_{E_n}f_n\dd{\mu}\geq c\int_{E_n} s\dd{\mu}$\par
En passant é la limite sur $n$ on a donc $\alpha\geq c\int_{X}s\dd{\mu}$ ce qui permet de conclure. 
\prop{Lemme de Fatou.} Soit $f_n : X\to\segment{0}{\infty}$ une suite de fonctions mesurables. Alors~:\par
$$\int \liminf f_n\dd{\mu}\leq \liminf\int f_n\dd{\mu}$$ 
\theoreme{Convergence dominée.}
Soit $f_n:X\to\cc$ mesurables telles qu'il existe $g\in\lll^1(\mu)$ avec $\va{f_n}\leq\va{g}$. On suppose que 
$f_n$ converge simplement vers $f$. Alors $f$ est intégrable et~:\par
$$\limn \int \va{f_n - f}\dd{\mu} = 0$$\par
On peut permuter limite et intégrale.
\rmq C'est un bel exemple de la puissance de l'intégrale de Lebesgue car sans cette théorie, le résultat est trés 
difficilement prouvable.
\dem $f\in\lll^1$ par passage é la limite. Ensuite, $\va{f_n -f}\leq 2g$. On applique le lemme de Fatou é la fonction 
$2g - \va{f_n -f}$. Ensuite on obtient\newline
$\limsup \int \va{f_n-f}\dd{\mu}\leq 0$ ce qui donne le résultat.

\theoreme{} Soit $f:X\to\segment{0}{\infty}$ une application mesurable. Alors~:\par
$$\phim(E) = \integrale{E}{}{f\dd{\mu}}$$\par
définit une mesure sur $\mmm$ (la tribu de $\mu$) et pour toute fonction mesurable de $X$ dans $\segment{0}{\infty}$~:\par
$$\integrale{X}{}{g\dd{\phim}} = \integrale{X}{}{fg\dd{\mu}}$$\par
On écrit la derniére inégalité sous la forme $\dd{\phim} = f\dd{\mu}$. La demonstration utilise le théoréme de la 
convergence monotone.

\spartie{Construction de mesures.}
\sspartie{Mesure de Lebesgue.}
La mesure de Lebesgue est la mesure la plus utile en théorie de l'intégration puisque l'intégrale par rapport é cette mesure d'une
fonction continue coincide avec l'intégrale de Riemann. C'est l'unique mesure qui vérifie $\lambda(\intervalle{a}{b}) = b-a$. Nous
avons prouvé son unicité mais pour établir son existence il faut beaucoup de place. D'un cété on peut la construire é partir du
théoréme de Riesz et de l'intégrale de Riemann, comme dans $\lbrack 1\rbrack$. Mais si cette construction est générale (car on a 
directement le cas de ${\rr}^d$ et la régularité) elle est coéteuse d'un point de vue théorique. D'un autre cété on peut la construire
en dimension 1 à partir des mesures extérieures. C'est ce qu'on va faire rapidement ici~:\par
\definition{Mesure extérieure.}
Une application $\lambda^* : \parties{X}\to\segment{0}{\infty}$ est une mesure extérieure si\par
\begin{itemize}
\item{} $\lambda^*(\emptyset) = 0$
\item{} $A\subset B\implique \lambda^*(A)\leq \lambda^*(B)$
\item{} $\forall A_k\in\parties{X}, \lambda^*\bigl(\bigcup\limits_{k\in \nn} A_k\bigr)\leq \serie{k\in \nn}{}{\lambda^*(A_k)}$
\end{itemize}\par
\definition{$\lambda^*$-mesurabilité.}
Une partie $A\subset X$ est $\lambda^*$-mesurable ssi~:\par
$$\forall B\subset X,\quad \lambda^*(B) = \lambda^*(B\cap A) + \lambda^* (B\cap A^c)$$\par
On note $\mmm(\lambda^*)$ l'ensemble des parties $\lambda^*$-mesurables.
\prop{} $\mmm(\lambda^*)$ est une tribu, qui contient toutes les parties telles que $\lambda^*(A) = 0$ et la restriction
de $\lambda^*$ é $\mmm(\lambda^*)$ est une mesure.\par
Pour construire la mesure de Lebesgue, il suffit alors de poser la bonne mesure extérieure, é savoir~:\par
$$\lambda^*(A) = \inf\lbrace \serie{i\in \nn}{}{(b_i-a_i)} : A\subset \bigcup\limits_{i\in \nn} \intervalle{a_i}{b_i}\rbrace$$\par
On vérifie alors le~:\par
\theoreme{} \begin{itemize}
\item{}  $\lambda^*$ est une mesure extérieure sur $\rr$.
\item{} La tribu $\mmm(\lambda^*)$ contient les boréliens.
\item{} Pour tous $a\leq b$, $\lambda^*(\segment{a}{b}) = \lambda^*(\intervalle{a}{b}) = b-a$
\end{itemize}
On construit de méme la mesure de Lebesgue en dimension $d$. Cette mesure est compléte (tout sous ensemble d'un ensemble mesurable de
mesure nulle est mesurable et de mesure nulle) et la tribu $\mmm(\lambda^*)$ coincide avec la complétée de la tribu borélienne.
\theoreme{Mesure de Lebesgue.} La mesure $\lambda$ posséde les propriétés suivantes~:
\begin{itemize}
\item{} $\lambda(\Pi_i\intervalle{a_i}{b_i}) = \Pi_i(b_i-a_i)$ pour tous $a_i, b_i$.
\item{} $\lambda$ est réguliére~: pour tout $A$ mesurable on a~: \par
\begin{tabular}{lcl}
$\lambda(A)$& $=$&  $\inf\lbrace\lambda(U) : U \mathrm{ouvert}, A\subset U\rbrace$\\
            & $=$&  $\sup\lbrace\lambda(K) : K\mathrm{compact}, K\subset A\rbrace$
\end{tabular}
\item{} $E\in \mmm$ ssi il existe $A$ un $F_{\sigma}$ et $B$ un $G_{\delta}$ tel que $A\subset E\subset B$ et 
         $\lambda(B-A) = 0$
\item{} $\lambda$ est invariante par translation.
\item{} Si $\mu$ est une mesure de Borel réguliére et invariante par translation alors $\mu = c \lambda$.
\item{} $\forall T\in \lll({\rr}^k), \exists\Delta(T)$ tel que $\lambda(T(E)) = \Delta(T)\lambda(E)$.
\end{itemize}
L'invariance par translation est donc cruciale puisqu'elle caractérise la mesure de Lebesgue parmi les mesures 
réguliéres. On ne prouvera pas ces propriétés dans un souci de concision.
\medskip
\rmq Il est facile de voir, par un argument de cardinalité, que tous les ensembles de $\mmm$ ne sont pas boréliens.
En effet $\bbb(\rr)$ est engendré par une famille dénombrable (les pavés é coordonnées rationnelles) et on peut montrer
qu'alors cette tribu a méme cardinal que $\rr$, soit $c$ (pour ``continu''). Par contre tous les sous ensembles du Cantor
sont mesurables (car il est de mesure nulle et que $\lambda$ est compléte). Or il a le cardinal du continu et donc $2^c$
parties, ce qui montre qu'il y a autant d'ensembles mesurables que de parties de $\rr$.\par 
Il est moins évident qu'il existe des ensembles non-mesurables au sens de Lebesgue. En admettant l'axiome du choix, 
c'est pourtant le cas. Si on considére le groupe quotient de $\rr$ par $\qq$ et qu'on pose $\mathbb{H}$ un ensemble
contenant un et un seul représentant de chaque classe d'équivalence. Alors $\rr = \cup_{x\in \qq} \mathbb{H}+x$
ce qui montre que $m(\mathbb{H}) > 0$. Mais d'un autre coté $\cup \mathbb{H} + x \subset \segment{0}{2}$ o\`u la 
réunion est prise sur tous les rationnels de $\segment{0}{1}$ ce qui montre $m(\mathbb{H}) = 0$. Cette absurdité
montre que $\mathbb{H}$ n'est pas mesurable (il n'est donc pas non plus borélien). Cependant, on a utilisé l'axiome du
choix.\par
\sspartie{Mesure Produit.}
\definition{Tribu produit.}
Soient $(E,\aaa)$ et $(F,\bbb)$ deux espaces mesurables. On munit $E\times F$ de la tribu produit~:\par
$$A\otimes B = \sigma(A\times B : A\in \aaa, B\in \bbb)$$\par
On peut de méme définir $\aaa_1\otimes\ldots \otimes \aaa_n = \sigma(A_1\times\ldots A_n; A_i\in \aaa_i)$ et on a associativité
dans le produit de tribu.
\definition{Section.}
Si $C\subset E\times F$, $x\in E, y\in F$ on pose~:\par
$$C_x = \lbrace y\in F : (x,y)\in C\rbrace$$\par
$$C^y = \lbrace x\in E : (x,y)\in C\rbrace$$\par
Si $f$ est une fonction de $E\times F$ on pose $f_x(y) = f(x,y)$ et $f^y(x) = f(x,y)$.\par
\prop{} 
\begin{itemize}
\item{} Si $C\in \aaa\otimes\bbb$ alors $C_x\in \aaa$ et $C^y\in \bbb$ pour tous $x,y$.
\item{} Si $f$ est mesurable pour $\aaa\otimes \bbb$ alors $f_x$ est mesurable pour $\bbb$ et $f^y$ l'est pour $\aaa$.
\end{itemize}
\dem é $x$ fixé on pose $\ccc_x = \lbrace C\in \aaa\otimes\bbb : C_x\in \bbb\rbrace$ et on vérifie que c'est une tribu
qui contient les pavés mesurables. Pour l'autre on écrit $f_x^{-1}(D) = \lbrace y\in F : (x,y)\in f^{-1}(D) = (f^{-1}(D))_x$\par
\theoreme{Mesure produit.} Soient $\mu, \nu$ deux mesures $\sigma$-finies.
\begin{itemize} 
\item{} Il existe une unique mesure $\mu\otimes\nu$ sur $\aaa\otimes\bbb$ telle que $\mu\otimes\nu(A\times B) = \mu(A)\nu(B)$.
\item{} Pour tout $C\in \aaa\otimes \bbb$,\par
$\mu\otimes \nu(C) = \integrale{E}{}{\nu(C_x)\mu(\mathrm{d}x)} = \integrale{F}{}{\mu(C^y)\nu(\mathrm{d}y)}$
\end{itemize}\par
\dem L'unicité provient du lemme de classe monotone. Pour l'existence on pose~:\par
$\ggg = \lbrace C\in \aaa\otimes\bbb : x\to\nu(C_x)\;\mathrm{soit}\;\aaa\mathrm{-mesurable}\rbrace$ et on vérifie que c'est une classe monotone qui contient les pavés mesurables. Le lemme
de classe monotone assure $\ggg = \aaa\otimes\bbb$ et on vérifie alors que $m(C) = \int_E \nu(C_x)\mu(\mathrm{d}x)$ définit une 
mesure.\par
On rappelle maintenant deux théorémes fondamentaux sur les mesures produits.
\theoreme{Fubini-Tonnelli.} Soient $\mu,\nu$ deux mesures $\sigma$-finies et soit $f : E\times F\to \segment{0}{\infty}$ une fonction 
mesurable. Alors on a~:
\begin{itemize}
\item{} $x\to \integrale{F}{}{f(x,y)\nu(\mathrm{d}y)}$ et $y\to \integrale{E}{}{f(x,y)\mu(\mathrm{d}x)}$ sont respectivement $\aaa$ et 
$\bbb$-mesurables. 
\item{} $\integrale{E\times F}{}{f\mathrm{d}\mu\otimes\nu} = \integrale{E}{}{\Bigl(\integrale{F}{}{f(x,y)\nu(\mathrm{d}y)}\Bigr)\mu(\mathrm{d}x)} = \integrale{F}{}{\Bigl(\integrale{E}{}{f(x,y)\mu(\mathrm{d}x)}\Bigr)\nu(\mathrm{d}y)}$
\end{itemize}\par
\dem Soit $C\in \aaa\otimes \bbb$. Si $f = 1_C$, on a déjé vu que $x\to \nu(C_x)$ est $\aaa$-mesurable. On en déduit par linéarité que
le résultat est vrai pour toute fonction étagée. Si $f$ esy positive, on peut écrire $f = \limncroiss f_n$ o\`u les $f_n$ sont étagées
positives. On utilise alors le théoréme de convergence monotone pour conclure.
\theoreme{Fubini-Lebesgue.} Soit $f\in \lll^1(E\times F, \aaa\otimes\bbb, \mu\otimes\nu)$. Alors~:\par
\begin{itemize}
\item{} $\mu$-pp, $f_x$ est dans $\lll^1(F,\bbb,\nu)$ et $\nu$-pp, $f^y$ est dans $\lll^1(E,\aaa,\mu)$.
\item{} $x\to \integrale{F}{}{f(x,y)\nu(\mathrm{d}y)}$ et $y\to \integrale{E}{}{f(x,y)\mu(\mathrm{d}x)}$ sont respectivement dans 
$\lll^1(E,\aaa,\mu)$ et dans $\lll^1(F,\bbb,\nu)$.
\item{} $\integrale{E\times F}{}{f\mathrm{d}\mu\otimes\nu} = \integrale{E}{}{\Bigl(\integrale{F}{}{f(x,y)\nu(\mathrm{d}y)}\Bigr)\mu(\mathrm{d}x)} = \integrale{F}{}{\Bigl(\integrale{E}{}{f(x,y)\mu(\mathrm{d}x)}\Bigr)\nu(\mathrm{d}y)}$
\end{itemize}
\rmq Les fonctions de $\lll^1$ sont définies é un ensemble de mesure nulle prés car en fait l'espace intéressant est $L^1$ qui est
le quotient de $\lll^1$ par la relation d'équivalence $\sim$ d'égalité $\mu$-pp. En effet $L^1$ (et tous les $L^p$) est un espace
métrique complet, contrairement é $\lll^1$ qui n'a pas de topologie intéressante. C'est pourquoi il n'est pas génant que les fonction
du théoréme ne soient définies que presque partout.
\dem On commence par appliquer le théoréme précédent é $\va{f}$ pour conclure é l'intégrabilité de $f^y$ et $f_x$. Pour ce qui est
des intégrales partielles, on écrit $f = f^+-f^-$ et on applique le théoréme précédent é $f^+$ et $f^-$.\par
Il faut noter que l'hypothése $f\in L^1(\mu\otimes\nu)$ est fondamentale puisque par exemple $f(x,y) = 2e^{-2xy}-e^{-xy}$ définie
sur $\intervalle{0}{+\infty}\times\intervalleg{0}{1}$, est intégrable selon $x$, selon $y$ et quand on calcule les deux intégrales 
croisées elles sont différentes.
\spartie{Mesures complexes.}
On va maintenant parler des mesures complexes, dans le but de démontrer le théoréme de Radon-Nykodim, sans doute le théoréme
le plus puissant de toute la de l'intégration qui permet par exemple de montrer l'existence d'espérance conditionnelle. Nous ne 
démontrons donc aucun des résultats intermédiaires mais nous démontrerons Radon-Nykodim, en admettant toutefois certains résultats comme
l'isomorphisme canonique entre un hilbert et son dual topologique ou d'autres propriétés qui ne serviraient ensuite. 
\definition{Mesure complexe} C'est une application d'une tribu dans $\cc$ qui vérifie la propriété suivante~:\par
$$\mu(E) = \serie{i=1}{+\infty}{\mu(E_i)}$$\par
Pour toute partition dénombrable $(E_i)$ de $E$.\par
\rmq Une mesure complexe ne prend donc jamais de valeur infinie, contrairement é une mesure positive. De plus, 
la valeur de la série ne dépendant pas de l'ordre de sommation des termes, elle est absolument convergente. On peut
alors associer é $\mu$ une mesure positive, la plus petite mesure positive qui domine $\mu$, appelée variation totale~:\par
\definition{Variation totale} Soit $\mu$ une mesure complexe. On pose~:\par
$$\va{\mu}(E) = \sup \serie{}{}{\va{\mu(E_i)}}$$\par
le sup étant pris sur toutes les partitions de $E$.\par
\prop{} La variation totale d'une mesure complexe est une mesure positive telle que $\va{\mu}(X)<+\infty$.
\definition{Absolue continuité.} Soit $\mu$ une mesure positive sur $\mmm$ et $\lambda$ une mesure positive ou 
complexe. Si $\mu(E) = 0$ implique $\lambda(E) = 0$ on dit que $\lambda$ est absolument continue par rapport é $\mu$
et on écrit~:\par
$$\lambda\ll \mu$$\par
Si $\lambda_1, \lambda_2$ sont portées par deux ensembles disjoints on dit qu'elles sont mutuellement singuliéres et
on écrit~:\par
$$\lambda_1\bot \lambda_2$$\par
\theoreme{Radon-Nikodym}
Soit $\mu$ une mesure positive $\sigma$-finie sur $\mmm$ dans $X$ et $\lambda$ une mesure complexe sur $\mmm$.
\begin{itemize}
\item{a)} $\exists !\lambda_a, \lambda_s$ mesures complexes sur $\mmm$ telles que~:\par
$$\lambda = \lambda_a + \lambda_s,\; \lambda_a\ll \mu,\; \lambda_s\bot\mu$$\par
Si $\lambda$ positive et finie, $\lambda_a,\lambda_s$ aussi et sont mutuellement singuliéres.
\item{b)}$\exists ! h\in L^1(\mu)$ telle que $\forall E\in \mmm$~:\par
$$\lambda_a(E) = \integrale{E}{}{h\dd{\mu}}$$
\end{itemize}
La $\sigma$-finitude est le fait que $X$ soit réunion dénombrable d'ensembles de mesure finie.
\rmq Ce théoréme n'est utilisé en probabilités qu'avec des mesures réelles appelées mesures ``signées''. Mais la 
démonstration étant la méme, autant le présenter dans son cadre complexe. 
\dem Supposons $\lambda$ positive et bornée. On peut s'y ramener en décomposant partie réelle et imaginaire. On 
introduit une fonction $w\in L^1(\mu)$ comprise strictement entre $0$ et $1$ (C'est lé qu'intervient la 
$\sigma$-finitude) et posons $\dd{\phim} = \dd{\lambda}+ w\dd{\mu}$. On aura alors pour toute fonction mesurable~:
\begin{equation}
\int_X f\dd{\phim} = \int_X f\dd{\lambda} + \int_X fw\dd{\mu}
\end{equation}
Si $f\in L^2(\phim)$ l'inégalité de Cauchy-Schwarz montre que $f\to \int_X f\dd{\lambda}$ est une forme linéaire
bornée sur $L^2(\phim)$. On utilise alors un théoréme sur les espaces de Hilbert pour trouver $g\in L^2(\mu)$
telle que pour toute $f\in L^2(\phim)$ on ait~: $\int_X f\dd{\lambda} = \int_X fg\dd{\phim}$. Mais en faisant 
$f =\chi_E$ dans $(1)$, en utilisant $0\leq \lambda\leq \phim$ on a~:
\begin{equation}
0\leq \frac{1}{\phim(E)}\int_E g\dd{\phim}
\end{equation}
et un théoréme montre alors que $0\leq g\leq 1$ $\phim$-presque partout. En redéfinissant $g$ de maniére arbitraire
sur un ensemble de mesure nulle on peut donc la supposer dans $\segment{0}{1}$ pour tout $x$. On récrit $(1)$ sous la
forme~:
\begin{equation}
\int_X (1-g)f\dd{\lambda} = \int_X fgw\dd{\mu}
\end{equation}
On pose alors $A = \lbrace x| 0\leq g(x)<1\rbrace$ et $B = \lbrace x| g(x) = 1\rbrace$. On définit ensuite~:\par
\centerline{$\lambda_a(E) = \lambda (A\cap E)$ et $\lambda_s (E) = \lambda (B\cap E)$}\par
On fait alors successivement $f = \chi_B$ et $f = (1+g+g^2 +\ldots g^n)\chi_E$ dans l'égalité $(3)$ pour arriver au 
résultat.\par








