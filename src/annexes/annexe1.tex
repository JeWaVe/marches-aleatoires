\appendix
\section{Programmes en Caml}
\uline{Résolution du problème de Dirichlet discret en caml et représentation graphique des solutions.}\par
On se place en dimension $2$. Les programmes qui suivent fournissent une résolution numérique approchée du problème
discret tel qu'il est traité dans la partie $2$. Dans un souci de concision, nous ne faisons pas figurer des programmes
élémentaires comme la fabrication d'une matrice, la détermination de son maximum ou autres tâches qui sont
en fait de simples boucles for.\par
Le caml-toplevel à lancer est \textit{camllight camlgraph}.
Il faut commencer par appeler les bibliothèques graphiques de J.M. Quercia ce qu'on fait avec les commandes~:\par
{\begin{verbatim}
#open "graphx";;
load_object "graphx";;
\end{verbatim}}
\uline{Calcul d'une valeur approchée de $\mathbb{E}_{(x,y)}(f(B_T))$~:}\par
{\begin{verbatim}
let dirichlet ouvert f x y n =
  let m = ref 0. in
    for i = 1 to n do
      let a = ref x and b = ref y in
	while ouvert !a !b do
	  let u = random__int 4 in
	    if u = 0 then a := !a + 1
	    else if u = 1 then a := !a - 1
	    else if u = 2 then b := !b + 1
	    else b := !b - 1
	done;
	m := !m +. (f !a !b );
    done;
    !m /. (float_of_int n);;
\end{verbatim}}
\uline{Calcul de la taille de l'ouvert~:}\par
{\begin{verbatim}
let borne_ouvert ouvert =
  let t1 = ref 0 in
    for i = 0 to 639 do
      for j = 0 to 639 do
        if (ouvert i j) && (i > !t1) then t1 := i
        else if (ouvert i j) && (j > !t1) then t1 := j
      done;
    done;
    !t1 + 1;;
\end{verbatim}}

\uline{Frontière de l'ouvert~:}\par
{\begin{verbatim}
let frontiere ouvert x y =
  if not (ouvert x y) then
    if (ouvert (x+1) y)||(ouvert (x-1) y)||(ouvert x (y+1))||(ouvert x (y-1)) then true
    else false
  else false;;
\end{verbatim}}

\uline{Calcul de la matrice des températures approchées~:}
{\begin{verbatim}
let solution ouvert f n =
  let t1 = borne_ouvert ouvert in
    let m = mat_float (t1+1) in(*fabrication d'une matrice de flottants*)
    for x = 0 to t1 do
      for y = 0 to t1 do
	if ouvert x y then
	  m.(x).(y) <- dirichlet ouvert f x y n
	else if frontiere ouvert x y then
	  m.(x).(y) <- 0.
	else m.(x).(y) <- 100.
      done;
    done;
    m;;
\end{verbatim}}

\uline{Affectation des couleurs~:}\par
{\begin{verbatim}
let couleur i =
  match i with
    |0 -> black |1 -> red  |2 -> yellow |3 -> green
    |4 -> blue  |5 -> cyan |6 -> white;;
\end{verbatim}}

\uline{Représentation graphique d'une matrice de flottants~:}\par
{\begin{verbatim}
let graphe_mat m =
  clear_graph();
  let t1 = vect_length m.(0) in
  let u = max m in(*détermination du maximum de m*)
    for i = 0 to (t1 - 1) do
      for j = 0 to (t1 - 1) do
	let c = int_of_float (7. *. (m.(i).(j) /. (u +. 1.))) in
	  set_color (couleur c);
	  fill_rect {x=i; y=j} 1 1;
      done;
    done;;
\end{verbatim}}

\uline{Le programme terminal~:}\par
{\begin{verbatim}
let graphe_solution ouvert f n =
  open_graph " 640x640";
  let m = solution  ouvert f n in
    graphe_mat m;;
\end{verbatim}}
{\begin{verbatim}
graphe_solution :
 (int -> int -> bool) -> (int -> int -> float) -> int -> unit = <fun>
\end{verbatim}}

