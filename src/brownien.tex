\documentclass[a4paper, 10pt]{article}
\usepackage[utf8]{inputenc}
\usepackage[francais]{babel}
\usepackage[T1]{fontenc}
\usepackage{color,amsmath,amssymb}
\usepackage{pst-all}
\usepackage{graphicx}
\usepackage{ulem}
\usepackage{maths}
\usepackage{hyperref}
\sloppy

% déclarations préliminaires
%
\begin{document}
\pagestyle{plain}

%
% Document à proprement parler
%
% Bibliographie
%
\nocite{*}
\bibliographystyle{plain}                   
%
%
\ 
\vspace{120pt}\par
\centerline{%\renewcommand{arraystretch}{2}
\begin{tabular}{c}
   \huge{Processus aléatoires}\\
\vspace{2pt}\\
   \huge{et}\\
\vspace{2pt}\\
   \huge{Problème de Dirichlet.}
\end{tabular}}
%
%


\newpage \centerline{\Large{\textbf{Introduction~:}}}\par\vspace{4ex}
Dans un condensateur, le potentiel électrostatique vérifie $\Delta V = 0$ ainsi que des 
conditions aux limites sur les armatures. Le problème de Dirichlet consiste à montrer
l'existence et l'unicité du potentiel dans de telles conditions. On se donne donc un
ouvert connexe borné $D$ de ${\rr}^d$, ainsi qu'une fonction continue $V_0$ sur la 
frontière $\partial D$ de $D$. $D$ représente l'espace vide inter-armatures, $\partial D$ 
les armatures et $V_0$ le ``potentiel aux bornes''. Il s'agit alors de montrer l'existence
et l'unicité de $V$ tel que $\Delta V = 0$ sur $D$ et $V=V_0$ sur $\partial D$.
Cette généralisation mathématique permettra d'étendre les résultats obtenus à la diffusion
thermique ou n'importe quel problème physique faisant intervenir des fonctions 
harmoniques.\par
L'unicité est facile à prouver en utilisant le principe du maximum. La partie
difficile du problème de Dirichlet est donc l'existence. On peut l'obtenir en minimisant
la fonctionnelle $V\to \int_D \norme{\grad{V}}^2$ mais on a alors besoin d'espaces de
Sobolev, ce qui est inaccessible à notre niveau. Nous avons choisi une approche
probabiliste, ce qui permet une construction explicite de la solution.\par
Afin de simplifier l'étude, nous nous contentons dans un premier temps du problème 
discret. Ainsi, dans une première partie, après avoir introduit des notions et théorèmes
de probabilité, nous étudions les marches aléatoires à temps discret. Nous verrons en 
effet que celles-ci jouent un rôle crucial dans la résolution du problème de Dirichlet
discret --- c'est-à-dire sur le réseau ${\zz}^d$. De plus nous retrouvons avec des outils 
élémentaires (formule de Stirling), le théorème central limite dans le cas des marches aléatoires.
\par
La deuxième partie est consacrée à sa résolution explicite. Dans cet esprit, nous introduisons les chaînes de Markov. 
Cette partie est la plus théorique. Nous définissons la chaîne de Markov sur un espace dénombrable, dont
nous admettons l'existence. Ensuite, nous démontrons la propriété de Markov, ce qui permet
de trouver une solution au problème de Dirichlet.\par
La troisième partie est consacrée à la résolution informatique du cas discret et du cas continu. 
Nous introduisons alors le mouvement brownien qui permet de trouver une 
solution dans le cas continu de même que les marches aléatoires le permettaient dans le cas 
discret. De même, nous avons réalisé des programmes en Caml qui permettent de visualiser des 
solutions approchées. Enfin, nous étudions la complexité moyenne de notre algorithme et nous la 
comparons à d'autres méthodes de résolution numérique.\par
Toutefois, étant donné la difficulté des outils mis en jeu relativement à notre niveau,
les résultats les plus difficiles de probabilité ou de théorie de l'intégration, comme
le théorème d'arrêt pour les martingales, l'existence de Brownien ou de la chaîne de 
Markov canonique sont admis. Les résultats plus abordables sont par contre démontrés.

\partie{Marches aléatoires.}
\spartie{Théorème importants.}
Pour commencer, rappelons des résultats utiles de probabilité.
\definition{Classe monotone.}
Un sous ensemble $\mmm$ de $\parties{E}$ est une classe monotone lorsque~:
\begin{itemize}
\item[(i)]{} $E\in \mmm$\par
\item[(ii)]{} Si $A, B\in \mmm$ alors $B\cap A^c\in \mmm$\par
\item[(iii)]{} Si $A_n$ est une suite croissante d'éléments de $\mmm$, $\mmm$ contient 
$\bigcup\limits_{n\in\nn} A_n$.
\end{itemize}\par
Ainsi, il est évident qu'une tribu est une classe monotone. Une intersection quelconque de
classes monotones reste une classe monotone, ainsi, comme pour les tribus, on peut définir la 
classe monotone engendrée par une partie $\ccc$ de $\parties{E}$ et on la note $\mmm(\ccc)=
\bigcap\limits_{\ccc\subset M}M$.
\theoreme{Lemme de Classe monotone.} Si $\ccc\subset\parties{E}$ est stable par intersections 
finies alors $\mmm(\ccc) = \sigma(\ccc)$ (la tribu engendrée par $\ccc$), ou de manière équivalente, $\mmm(\ccc)$ est une tribu.
\dem Il suffit en fait de montrer que $\mmm(\ccc)$ est stable par intersections finies. En effet
une fois qu'on sait cela les vérifications à faire pour les axiomes d'une tribu sont faciles.
On fixe donc $A\in \ccc$ et on pose $\mmm_1 = \ens{B\in\mmm(\ccc)|A\cap B\in\mmm(\ccc)}$. On a
bien sur $\ccc\subset \mmm_1$. On vérifie ensuite que c'est une classe monotone (preuve omise pour
gagner en concision, elle n'est pas difficile). La propriété de minimalité de $\mmm(\ccc)$ montre
alors que $\mmm_1$ contient $\mmm(\ccc)$. On a donc~:\par
$$\forall A\in\ccc, B\in\mmm(\ccc); A\cap B\in\mmm(\ccc)$$\par
On fixe alors $B\in \mmm(\ccc)$ et on pose $\mmm_2 = \ens{A\in\mmm(\ccc)|A\cap B\in\mmm(\ccc)}$.
On vient de montrer que $\ccc\subset\mmm_2$. Ensuite on vérifie que $\mmm_2$ est une classe 
monotone, ce qui conclut la preuve.\hfill\fbox{{}}\par
\proposition Soient $\bbb_1, \ldots, \bbb_n$ des sous-tribus de $\bbb$, tribu
sur laquelle on définit une mesure de probabilité $P$. On se donne pour tout $i$ une classe 
$\ccc_i\subset\bbb_i$, stable par intersections finies, contenant $\Omega$ (l'espace probabilisé)
et telle que $\sigma(\ccc_i) = \bbb_i$. De plus on suppose que 
$P(C_1\cap\ldots\cap C_n) = P(C_1)\ldots P(C_n)$ pour tous $C_i\in\bbb_i$. Alors les tribus 
$\bbb_i$ sont indépendantes.
\dem Par définition de l'indépendance, montrons que~: 
$\forall B_i\in\bbb_i,\quad P(B_1\cap\ldots\cap B_n) = P(B_1)\ldots P(B_n)$. 
On fixe $C_2,\ldots,C_n$ et on pose 
$\mmm_1 = \ens{B_1\in\bbb_1|P(B_1\cap C_2\cap\ldots\cap C_n) = P(B_1)P(C_2)\ldots P(C_n)}$.
On vérifie que $\ccc_1\subset \mmm_1$ et que $\mmm_1$ est une classe monotone. Le lemme de classe
monotone assure alors que $\mmm(\ccc_1) = \sigma(\ccc_1) = \bbb_1$. Et donc on a montré
que dans la propriété de l'énoncé on peut prendre $C_1$ dans $\bbb_1$ et pas seulement dans 
$\ccc_1$. On reproduit alors le raisonnement en fixant $B_1\in\bbb_1, C_i\in\ccc_i$ et
une récurrence montre le résultat voulu.\findem
\proposition On prend $\bbb_1,\ldots,\bbb_n$ des tribus indépendantes et on les regroupe en deux paquets
$\ddd_1 = \sigma(\bbb_1, \ldots\bbb_p)$, $\ddd_2 = \sigma(\bbb_{p+1},\ldots \bbb_n)$.
Alors $\ddd_1$ et $\ddd_2$ sont indépendantes.
\dem On prend $\ccc_1 = \ens{B_1\cap\ldots\cap B_p}$ et $\ccc_2 = \ens{B_{p+1}\cap\ldots\cap B_n}$.
On aplique ensuite la proposition précédente.\findem
\proposition Soit $(\bbb_n)_{n\in\nn}$ une famille infinie de tribus indépendantes. Alors pour tout
$p$, $\ddd_1 = \sigma(\bbb_1, \ldots, \bbb_p)$ et $\ddd_2 = \sigma(\bbb_{p+1},\ldots)$ sont 
indépendantes.
\dem On prend $\ccc_1 = \bbb_1$ et $\ccc_2 = \bigcup\limits_{k=p+1}^{+\infty}\sigma(
                                                     \bbb_{p+1},\ldots,\bbb_k)\subset\ddd_2$.
Et on applique le proposition 1.2 pour montrer que $\ccc_2$ est stable par intersections finies.
Ensuite, la proposition 1.1 donne le résultat.\findem
\definition{} Si $A_n$ est une suite d'évènements, on note 
$\limsup A_n = \bigcap\limits_{n=0}^{+\infty}\bigcup\limits_{k=n}^{+\infty}A_k$.
\theoreme{Lemme de Borel-Cantelli}
Soit $A_n$ une suite d'évènements.
\begin{itemize}
\item[(i)]{} Supposons $\serie{n=0}{+\infty}{P(A_n)}<+\infty$. Alors $P(\limsup A_n) = 0$ ou de
manière équivalente, presque tout $\omega$ n'appartient qu'à un nombre fini de $A_i$.
\item[(ii)]{} Si $\serie{n=0}{+\infty}{P(A_n)}=+\infty$ et que les $A_n$ sont indépendants, 
$P(\limsup A_n) = 1$ ou de manière équivalente, presque tout $\omega$ appartient à un nombre infini
de $A_i$.
\end{itemize}
\dem \begin{itemize}
\item[(i)]{} On a $\esp{\sum_n \un_{A_n}} = \sum_n P(A_n) <+\infty$. Donc $\sum_n \un_{A_n}<\infty$
$P$-p.s.
\item[(ii)]{} Remarquons déjà que l'hypothèse d'indépendance est indispensable puisque par exemple
en prenant $A_n = A$ pour tout $n$, avec $0<P(A)<1$. Ensuite
on fixe $n_0$ et $n\geq n_0$. Alors l'indépendance des $A_k$ (et donc des $A_k^c$) donne 
$P(\bigcap_{n_0}^n A_k^c) = \Pi_{k=n_0}^n P(A_k^c) = \Pi_{k=n_0}^{n}(1-P(A_k))$. Un résultat 
classique d'analyse montre alors que puisque la série des $P(A_k)$ diverge, ce produit converge 
vers $0$ quand $n$ tend vers l'infini. On a donc $P(\bigcap\limits_{k=n_0}^{\infty}A_k^c) = 0$.
On obtient alors $P(\bigcup\limits_{n_0 = 0}^{\infty}\bigcap\limits_{k=n_0}^{\infty}A_k^c) = 0$
puis, en passant au complémentaire, $P(\limsup A_n) = 1$.
\end{itemize}\par\findem
\theoreme{Loi du tout ou rien.} Soit $X_n$ une suite de v.a. indépendantes, à valeurs dans des 
espaces mesurables quelconques. On définit $\bbb_n = \sigma(X_k, k\geq n)$ et $\bbb_{\infty} = 
\bigcap\limits_{n=1}^{+\infty} \bbb_n$. Cette dernière tribu est grossière, i.e. $\forall B\in\bbb_{\infty}$, $P(B)\in\ens{0;1}$.
\dem On pose $\ddd_n = \sigma(X_k; k\leq n)$. D'après la proposition 1.3, $\ddd_n$ est indépendante
de $\bbb_{n+1}$ et donc de $\bbb_{\infty}$. Alors on a montré que pour tout $A$ de 
$\ccc = \bigcup\limits_{n=1}^{+\infty}\ddd_n$ et tout $B$ de $\bbb_{\infty}$, $P(A\cap B)=P(A)P(B)$.
Mais la classe $\ccc$ est stable par intersections finies et la proposition 1.1 montre que 
$\bbb_{\infty}$ est indépendante de $\sigma{(\ccc)}$. $\bbb_{\infty}$ est donc en particulier 
indépendante d'elle même et donc si $B \in \bbb_{\infty}$, $P(B) = P(B\cap B) = P(B)^2$, ce qui
n'est possible que si $P(B)\in\ens{0; 1}$.\findem
\spartie{Application aux marches aléatoires.}
\theoreme{Comportement de la marche aléatoire simple.} Soit $X_n$ une suite de v.a. indépendantes, 
de même loi $\mu$ telle que $P(X_n=1) = 1/2 = P(X_n= -1)$. Pour $n\geq 1$, on pose~:\par
$$S_n = X_1 + \ldots X_n$$\par
C'est la marche aléatoire simple sur $\zz$, issue de $0$. Alors~:\par
$$\mathrm{p.s.}\quad\sup_{n\geq 1}S_n=+\infty\quad\mathrm{et}\quad\inf_{n\geq 1} S_n=-\infty$$\par
Et en particulier, $S_n$ s'annule presque sûrement une infinité de fois.
\dem Commençons par montrer que pour tout $p\geq 1$, $P(A)=0$ si 
$A = \ens{-p\leq\inf S_n\leq\sup S_n\leq p}$.
On fixe alors $k> 2p$ et on remarque que $\bigcup_{j=0}^{\infty}\ens{X_{jk+1}=\ldots=X_{jk+k}=1}
\subset A^c$. En effet si l'on est dans l'un des ensembles de gauche, alors 
$S_{jk+k} = 2p + S_{jk}$. De trois choses l'une, si $S_{jk} <-p$, on a bien $\inf S_n <-p$.
Si $S_{jk} > p$, on a bien $\sup S_n>p$. Enfin, si $-p\leq S_{jk}\leq p$, $S_{jk+k}> p$.\par
Ensuite, on montre que l'ensemble $\bigcup_{j=0}^{\infty}\ens{X_{jk+1}=\ldots=X_{jk+k}=1}$
a une probabilité $1$. On pose en effet $A_j = \ens{X_{jk+1}=\ldots=X_{jk+k}=1}$. Les évènements
$A_j$ sont indépendants grâce à la proposition 1.3 et leur probabilité vaut $\frac{1}{2^k}$.
On a donc $\sum_j P(A_j) = +\infty$ et le lemme de Borel-Cantelli montre que presque tout $\omega$
appartient à une infinité de $A_j$, donc à $A$. On a donc~:\par
Pour tout $p\geq 0$, $P(\ens{-p\leq\inf S_n\cap\sup S_n\leq p}) = 0$.
En faisant tendre $p$ vers l'infini, $P(\ens{-\infty<\inf S_n\cap\sup S_n<+\infty}) = 0$.
Puis, en passant au complémentaire, $P(\ens{-\infty=\inf S_n\cup\sup S_n=+\infty}) = 1$.
D'o\`u, $P(\ens{-\infty=\inf S_n})+P(\ens{\sup S_n=+\infty}) \geq 1$.\par
Comme on a bien entendu que ces deux probabilités sont égales (par symétrie), elles sont
toutes deux strictement positives. Or l'évènement $\ens{\sup S_n = +\infty}$ est dans la tribu
asymptotique $\bbb_{\infty}$ puisque pour tout $k\geq 1$, $\ens{\sup S_n = +\infty} = 
\ens{\sup_{n\geq k}(X_k +\ldots + X_n) = +\infty}\in \bbb_k$ (ce qui traduit le fait que les 
premiers termes, en nombre fini, ne jouent pas de rôle). La loi du tout ou rien montre alors que 
$P(\ens{\sup S_n = \infty}) = 1$.\findem
On a donc montré que la marche aléatoire simple sur $\zz$ n'est pas bornée. En prenant une autre
loi, telle que $P(X_n=-1) = 1/4 = P(X_n = 1)$ et $P(X_n=0) = 1/2$, on aurait obtenu le même 
résultat. On peut ensuite étendre ce résultat en dimension supérieure. Ainsi, prenons la marche 
aléatoire simple sur ${\zz}^2$, de loi $P(X_n = \pm e_i)=1/4$, o\`u $e_i$ sont les deux vecteurs de 
la base canonique de ${\zz}^2$. En la projetant sur la droite $y=0$, on obtient une marche aléatoire
uni-dimensionnelle, de loi donnée ci-dessus, dont on sait qu'elle diverge presque surement. On peut 
donc énoncer le même théorème en dimension $2$~:\par
\theoreme{} Soit $X_n$ une suite de v.a. indépendantes, de même loi donnée par $P(X_n=(0,\pm1))=1/4$
et $P(X_n = (\pm 1, 0)) = 1/4$. Posons $S_n = X_1 +\ldots+X_n$ la marche aléatoire simple sur 
${\zz}^2$ issue de $0$. Alors~:\par
$$p.s.\quad\sup (S_n)_x=+\infty=\sup(S_n)_y\quad\mathrm{et}\quad\inf(S_n)_x=+\infty=\inf (S_n)_y$$
\par
O\`u $(S_n)_x$ (resp. $(S_n)_y$) représente la coordonnée selon $x$ (resp. selon $y$) de $S_n$.\par
\spartie{La convergence en loi de la marche aléatoire simple.}
Comme nous le verrons dans la troisième partie le mouvement Brownien est 
une ``limite'' de marches aléatoires. Cette convergence se fait au sens de la convergence en loi.
\definition{Convergence en loi.} Une suite $(X_n)$ de v.a. à valeurs dans ${\rr}^d$ converge en loi vers
$X$ lorsque pour toute $\phim$ de $\ccc_b({\rr}^d)$, $\esp{\phim(X_n)}\tendn\esp{\phim(X)}$.\par
Remarquons que dans cette défition, la limite n'est pas définie de manière unique et que les
v.a. peuvent être définies sur des espaces de probabilité différents.\par
\theoreme{théorème de la limite centrale.} Soit $X_n$ une suite de v.a. indépendantes, de même loi, de carré intégrables.
On pose $\sigma^2 = \var(X_1)$. Alors on sait que~:\par
$$\dfrac{1}{\sqrt{n}}(X_1 + \ldots+X_n - nE(X_1))\tendloi \nnn(0,\sigma^2)$$\par
o\`u $\nnn(0,\sigma^2)$ désigne la loi gaussienne centrée de variance $\sigma^2$, de densité
$p(x) = \dfrac{1}{\sqrt{2\pi}}e^{-x^2/(2\sigma^2)}$.\par
Nous admettrons également ce théorème dans sa généralité, car sa démonstration nécessite trop de théorie supplémentaire. 
Toutefois, nous allons le démontrer dans le cas de la marche 
aléatoire simple sur $\zz$ et ce avec des outils élémentaires.\par
Prenons une marche aléatoire $S_n = X_1 + \ldots + X_n$ o\`u les $X_k$ sont indépendantes et vérifient $P(X_i=1)=P(X_i=0) = 1/2$ et
montrons la propriété~:\par
$$\limn\dfrac{1}{2^n}\serie{\frac{n}{2}\leq k\leq\frac{n}{2}+\sqrt{n}}{}{C^k_n}=\sqrt{\dfrac{2}{\pi}}\integrale{0}{1}{e^{-2x^2}\dd{x}}$$
\par
Cela revient en fait à prendre pour $\phi$ (dans la convergence en loi) l'indicatrice du segment $\segment{0}{1}$.\par
Pour ce faire, on utilise la formule de Stirling $n!\sim \bigl(\frac{n}{e}\bigr)^n\sqrt{2\pi n}$~:\par
On écrit donc $C^k_n \sim \frac{1}{\sqrt{2\pi}}\sqrt{\frac{n}{k(n-k)}}\bigl(\frac{n}{k}\bigr)^n\bigl(\frac{k}{n-k}\bigr)^{n-k}$.
En effet, si $n$ tend vers l'infini, $k$ ``le suit'' s'il reste dans l'intervalle entier considéré. Examinons chacun des trois facteurs.
\begin{itemize}
\item[$\bullet$]{$\sqrt{\dfrac{n}{k(n-k)}}$}. On écrit $k = n/2+h$. Alors ce terme vaut $\sqrt{\dfrac{n}{\frac{n^2}{4}-h^2}}$.
Un D.L donne alors $\frac{2}{\sqrt{n}}\bigl(1+2\frac{h^2}{n^2}\bigr)\sim \frac{2}{\sqrt{n}}$. Soit~:\par
$$\sqrt{\dfrac{n}{k(n-k)}}\sim \dfrac{2}{\sqrt{n}}$$\par
\item[$\bullet$]{$\bigl(\frac{n}{k}\bigr)^n$} Avec la même astuce de notation, prenons le log de cette quantité.
$n\ln \bigl(\frac{n}{k}\bigr) = n\ln\bigl(\dfrac{n}{n/2 +h}\bigr) = n\ln 2 -n\ln(1+2h/n)$. On effectue alors un D.L à l'ordre 2 du log
pour obtenir $n\ln 2 -2h +2\frac{h^2}{n^2} + \petito{1}$. On a alors le droit de prendre l'exponentielle pour obtenir~:\par
$$\bigl(\frac{n}{k}\bigr)^n\sim 2^ne^{-2h}e^{2\frac{h^2}{n}}$$
\item[$\bullet$]{$\bigl(\frac{k}{n-k}\bigr)^{n-k}$} On fait de même, mais cette fois le calcul est plus long. Le log de cette 
quantité vaut $(n/2-h)\ln\bigl(\frac{n/2+h}{n/2 -h}\bigr)$, soit $(n/2 -h)\ln\left\lbrack(1+\frac{2h}{n}\bigr)\bigl(1+\frac{2h}{n}+\frac{4h^2}{n^2}+\petito{1/n}\bigr)\right\rbrack$. On simplifie alors cette expression sous la forme~:
$(n/2-h)\ln\lbrack 1 + \frac{4h}{n} + \frac{8h^2}{n^2} +\petito{1/n}\rbrack = (n/2-h)(4h/n+\petito{1/n}$, en utilisant le D.L
du logarithme en $1$. On obtient finalement $2h -4h^2/n^2 + \petito{1}$. En passant à l'exponentielle on a~:\par
$$\bigl(\frac{k}{n-k}\bigr)^{n-k}\sim e^{2h}e^{-4\frac{h^2}{n}}$$\par
\end{itemize}
En regroupant ces résultats, on a donc pour $k$ dans l'intervalle considéré~:\par
$$2^{-n}C^k_n= \sqrt{\dfrac{2}{\pi}}\frac{1}{\sqrt{n}} e^{-2\bigl(\frac{h}{\sqrt{n}}\bigr)^2}+\petito{1/\sqrt{n}}$$\par
Ce qui permet de conclure, en reconnaissant une somme de riemann.
On a donc montré le théorème central limite dans le cas particulier d'une marche aléatoire simple
sur $\zz$. Il est intéressant de le vérifier informatiquement, et de constater la convergence
en loi. (c-f annexe pour le programme réalisant cela).
\centerline{\includegraphics*[width=20cm,height=2cm]{images/gaussienne.ps}}\par
et l'allure gaussienne de la loi apparaît clairement. 
\partie{Chaînes de Markov et problème de Dirichlet discret.} 
\spartie{Définitions et premières propriétés.}
\definition{Loi conditionnelle.}
Soient $X,Y$ deux v.a. à valeurs dans $(E,\eee)$. On appelle loi conditionnelle de $Y$ sachant 
$X$ toute application $\nu~:E\times\eee\to\segment{0}{1}$ telle que~:\par
\begin{itemize}
\item[(i)]{} Pour tout $x$, $A\to\nu(x,A)$ est une mesure de probabilité sur $\eee$.
\item[(ii)]{} Pour tout $A$, $x\to\nu(x,A)$ est une application mesurable.
\item[(iii)]{} Pour toute fonction $h$ mesurable positive, on a~:\par
$$\espc{h(Y)}{X} = \int_{E} h(y)\nu(X,\dd{y})$$
\end{itemize}\par
Ainsi, on aura $P(Y\in A|X) = \nu(X,A)$ p.s. et tous les calculs d'espérance conditionnelle sont 
considérablement facilités.\par
\rmq Une application qui vérifie $(i)$ et $(ii)$ est appelée probabilité de transition de $E$ dans $E$.\par
Désormais, on supposera qu'on se place sur un espace $E$ dénombrable, muni de la tribu de ses parties.
\definition{Matrice stochastique.}
Une matrice stochastique sur $E$ est une famille $(Q(x,y))_{(x,y)\in E^2}$ de réels telle que~:
\begin{itemize}
\item[(i)]{} $0\leq Q(x,y)\leq 1$ pour tout $x,y$.
\item[(ii)]{} $\sum\limits_{y\in E} Q(x,y) = 1$ pour tout $x$.
\end{itemize}
La notion précédente et celle de probabilité de transition sont équivalentes. En effet, si $Q$ est une 
matrice stochastique, $\nu(x,A)=\sum_{y\in A}Q(x,y)$ est une probabilité de transition de $E$ dans $E$
et si $\nu$ est une probabilité de transition, $Q(x,y) = \nu(x,\ens{y})$ est une matrice stochastique.\par
On définit par récurrence les puissances de $Q$ comme $Q_1 = Q$ et $Q_{n+1}(x,y)=\sum_z Q_n(x,z)Q(z,y)$. On
vérifie facilement que $Q_n$ est une matrice stochastique (cette définition correspond au produit
matriciel classique si $E$ est fini).\par
Pour toute fonction $f~:E\to {\rr}_+$ et $Q$ une matrice stochastique sur $E$,
on définit $Qf(x) = \sum\limits_{y\in E}Q(x,y)f(y)$.
\definition{Chaîne de Markov.} Soit $Q$ une matrice stochastique et $X_n$ un processus aléatoire à valeurs
dans $E$. On dit que $X_n$ est une chaîne de Markov de matrice de transition $Q$ lorsque pour tout $n$, 
la loi conditionnelle de $X_{n+1}$ sachant $(X_0,\ldots,X_n)$ est $Q(X_n,y)$. De manière équivalente
cela signifie que~:\par
$$P(X_{n+1}=y|X_0=x_0,\ldots,X_n=x_n) = Q(x_n,y)$$\par
dès que $P(X_0=x_0,\ldots,X_n=x_n)>0$.\par
Intuitivement, le fait que la loi conditionnelle de $X_{n+1}$ ne dépende que de $X_n$ signifie que le 
processus ne se soucie pas du passé pour déterminer son comportement dans le futur mais seulement du présent.
Cette propriété est appelée propriété de Markov.\par
\rmq Soit $\xi_1,\ldots,\xi_n,\ldots$ une suite de v.a. indépentantes, à valeurs dans ${\zz}^d$, de même loi 
$\mu$. Alors $S_n = \xi_1 + \ldots + \xi_n$ est une chaîne de Markov, de matrice de transition $Q$ donnée par
$Q(x,y)=\mu(\ens{y-x})$.
\proposition Un processus $(X_n)$ à valeurs dans $E$ est une chaîne de Markov de matrice de transition $Q$ ssi~:\par
$$P(X_0=x_0,\ldots, X_n=x_n) = P(X_0=x_0)Q(x_0,x_1)\ldots Q(x_{n-1},x_n)$$\par
De plus, on a si $P(X_0=x_0)> 0$, $P(X_n=x_n|X_0=x_0)=P(X_0=x_0)Q_n(x_0,x_n)$
\dem Elle est facile en utilisant la définition d'une probabilité conditionnelle~: $P(A|B) = P(A\cap B)/P(B)$.
Pour la dernière assertion, on utilise que $Q_n(x_0,x_n) = \sum_{x_1,\ldots,x_n}Q(x_0,x_1)\ldots 
Q(x_{n-1},x_n)$.
\rmq Cette propriété montre que la loi du vecteur $(X_0,\ldots X_n)$ ne dépend que de $Q$ et de la loi 
initiale.
\proposition Soit $(X_n)$ une chaîne de Markov de matrice de transition $Q$.\par
$$\espc{f(X_{n+1})}{X_0,\ldots,X_n} = \sum_{y\in E} Q(X_n,y)f(y) = Qf(X_n)$$
\spartie{Chaîne de Markov canonique, propriété de Markov simple.}
\proposition{Admise} Soit $Q$ une matrice stochastique sur $E$. Alors il existe un espace de probabilité 
$(\Omega',\fff',P')$ sur lequel il existe, pour tout $x\in E$, un processus $(X_n^x)$ qui est une chaîne de 
Markov de matrice de transition $Q$ et tel que $X_0^x=x$ p.s.\par
Dans la suite, on posera $\Omega = E^{\nn}$. Il servira d'espace canonique sur lequel on définira nos chaînes
de Markov. Un élément $\omega\in \Omega$ est de la forme $(\omega_0,\omega_1,\ldots)$. On définit les 
applications coordonnées comme $X_n(\omega) = \omega_n$ et les opérateurs de translation comme
$\theta_n(\omega)=(\omega_n,\omega_{n+1},\ldots)$. On munit $\Omega$ de la plus petite tribu, notée $\fff$, 
qui rende mesurable les applications coordonnées. On remarque aussi que $\fff$ est la tribu engendrée par 
les cylindres, soit les ensembles $C = (\omega|\omega_0=x_0,\ldots,\omega_p=x_p)$.
De plus, on munit $\Omega$ de la filtration $\fff_n=\sigma(X_0,\ldots,X_n)$.
\proposition Soit $(B,\bbb)$ un espace mesurable et soit $\psi$ une application de $B$ dans $\Omega$. Alors $\psi$
est mesurable ssi $X_n\circ\psi$ l'est pour tout $n$.
\dem Il suffit de montrer que si chacune des $X_n\circ\psi$ sont mesurables, alors $\psi$ l'est aussi. Or
$\ens{A\in\fff : \psi^{-1}(A)\in\ggg}$ est une tribu qui par hypothèse contient tous les $X_n^{-1}(y),y\in E$
et rend donc les $X_n$ mesurables. Cette tribu contient alors naturellement $\fff$ toute entière par 
minimalité de $\fff$.
\rmq D'après ce résultat les opérateurs de translation sont mesurables.
\theoreme{Chaîne de Markov canonique.} 
Soit $Q$ une matrice stochastique sur $E$ et soit $x\in E$. Il existe une unique probabilité, notée
$\mathbb{P}_x$, sur $\Omega$, telle que sous $\mathbb{P}_x$, le processus des coordonnées $(X_n)$ définisse
une chaîne de Markov de matrice de transition $Q$, telle que $\mathbb{P}_x(X_0=x) = 1$.
\dem La proposition 2.3 permet de construire un espace $(\Omega',\fff',P')$ et un processus $(X_n^x)$ qui
est une chaine de Markov de matrice de transition $Q$ telle que $X_0^x=x$. On définit $\mathbb{P}_x$
comme la mesure image de $P'$ par l'application~:\par
$$\fonction{\Omega'}{\Omega}{\omega'}{((X_n^x(\omega'))_{n\in\nn}}$$\par
Cette application est mesurable grâce à la propriété précédente. On a $\mathbb{P}_x(X_0=x)=P'(X_0^x=x)=1$ et
$\mathbb{P}_x(X_0=x_0,\ldots,X_n=x_n) = P'(X_0^x=x_0,\ldots,X_n^x=x_n)$ par définition d'une mesure image.
Puis en utilisant la proposition 2.1, $= P'(X_0^x=x_0)Q(x_0,x_1)\ldots Q(x_{n-1},x_n) = \mathbb{P}_x(X_0=x_0)
Q(x_0,x_1)\ldots Q(x_{n-1,x_n}$ et en utilisant la réciproque de la proposition 2.1, on obtient bien que
sous $\mathbb{P}_x$, le processus des coordonnées est une chaîne de Markov de matrice de transition $Q$. Pour
l'unicité on constate que toutes les mesures satisfaisant à la propriété du théorème coincident sur les 
cylindres. Or les cylindres forment une classe stable par intersection finie qui engendre $\fff$ et le lemme
de classe monotone permet de conclure qu'elles sont égales.
\rmq On déduit immédiatement de la proposition 2.1 que $\mathbb{P}_x(X_n=x_n)=Q_n(x,y)$.\par
Il se pose le problème de la loi initiale pour montrer que l'on peut toujours ramener un problème sur une 
chaîne de Markov à la chaîne de Markov canonique. Pour cela, si $\mu$ et une mesure de probabilité sur $E$,
on pose $\mathbb{P}_\mu = \sum_x \mu(x)\mathbb{P}_x$. On vérifie que c'est une mesure de probabilité et
que sous $\mathbb{P}_\mu$, le processus des coordonnées forme une chaîne de Markov de matrice de transition
$Q$ et de loi initiale $\mu$.\par
On notera $\eesp{x}{\cdot}$ l'espérance sous $\mathbb{P}_x$.
\theoreme{Propriété de Markov simple.}
Soient $F$ et $G$ deux fonctions positives mesurables sur $\Omega$ et soit $n\geq 0$.
Si $F$ est $\fff_n$ mesurable, alors pour tout $x\in E$~:\par
$$\mathbb{E}_x\lbrack F\cdot G\circ\theta_n\rbrack=\mathbb{E}_x\lbrack F\mathbb{E}_{X_n}\lbrack 
                                                                                      G\rbrack\rbrack$$
\dem Il suffit de vérifier le résultat pour $F$ indicatrice d'un cylindre. C'est-à-dire
$F =\un_{\ens{X_0=x_0,\ldots X_n=x_n}}$. Pour cela, on commence par choisir 
$G = \un_{\ens{X_0=y_0,\ldots,X_p=y_p}}$. Alors $\eesp{y}{G} = \mathbb{P}_y(X_0=y_0,\ldots,X_p=y_p) = 
\un_{\ens{y_0=y}}Q(y_0,y_1)\ldots Q(y_{p-1},y_p)$ et donc~:\par 
$\eesp{x}{F\eesp{X_n}{G}} = \un_{\ens{x=x_0}}Q(x_0,x_1)\ldots
Q(x_{n-1},x_n)\un_{\ens{x_n=y_0}}Q(y_0,y_1)\ldots Q(y_{p-1},y_p)$.\par
D'autre part, on a $\eesp{x}{F\cdot G\circ\theta_n} = \mathbb{P}_x(X_0=x_0,\ldots,X_n=x_n,X_n=y_0,\ldots,
X_{n+p} = y_p)$. Soit en utilisant la proposition 2.1, $\un_{\ens{x=x_0}}Q(x_0,x_1)\ldots Q(x_{n-1},x_n)
\un_{\ens{x_n=y_0}}Q(y_0,y_1)\ldots Q(y_{p-1},y_p)$. Ce qui donne le résultat. Le lemme de classe monotone
montre que le résultat reste vrai pour toute fonction $G=1_A$ $A\in\fff$, ce qui permet de conclure.

\spartie{Problème de Dirichlet discret.}
Dans toute la suite, on fixe une matrice stochastique $Q$ qui sera en pratique celle de la marche aléatoire
simple dans ${\zz}^d$.\par
\definition{Fonction harmonique.} Une fonction $f~:E\to\rr_+$ est dite harmonique lorsque pour tout $x$ de 
$E$, on a~:\par
$$f(x)=Qf(x)$$\par
\definition{Problème de Dirichlet~:}
Soit $F$ une partie finie de ${\zz}^d$ On définit $\partial F = \ens{y\not\in F :\exists x\in F, \va{y-x}=1}$.
C'est la frontière de $F$. On note $\bar{F} = F\cup\partial F$ l'adhérence de $F$. On se donne une fonction
$g$ définie sur $\partial F$. Le problème consiste à montrer l'existence et l'unicité d'une fonction définie 
sur $\bar{F}$, harmonique sur $F$, et qui coincide avec $g$ sur $\partial F$.\par
\definition{} Pour $G$ une partie de ${\zz}^d$, on définit~:\par
$$T_G = \inf\ens{n\geq 0 : X_n\in G}$$\par
$T_G$ est un temps d'arrêt de la filtration $\fff_n$ car $\ens{T_G = n} = \ens{X_0\not\in G}\cap
\ldots\cap\ens{X_{n-1}\not\in G}\cap\ens{X_n\in G}$ et chacun de ces ensembles est $\fff_n$ mesurable.\par
On peut alors définir la variable aléatoire~: $X_{T_G}\un_{\ens{T_G<\infty}}$ comme $X_n$ si $T = n$ et $0$
si $T=\infty$.
\theoreme{} Soit $F$ un sous-ensemble non-vide de $E$ et posons $G$ le complémentaire de $F$. Soit 
$g~:G\to\rr_+$ une fonction bornée. Alors~:\par
\begin{itemize}
\item[(i)]{} La fonction~: $h(x) = \mathbb{E}_x\lbrack g(X_{T_G})\un_{\ens{T_g<\infty}}\rbrack$ est harmonique
sur $F$.
\item[(ii)]{} Si $T_G<\infty$ $\mathbb{P}_x$ p.s. pour tout $x$. Alors $h$ est l'unique fonction harmonique 
sur $E$ qui est harmonique sur $F$ et coincide avec $g$ sur $G$.
\end{itemize}
\dem On remarque que si $x\in F$, on a $\mathbb{P}_x$ p.s.~:\par
$$g(X_{T_G})\un_{\ens{T_G<+\infty}} = g(X_{T_G}\circ\theta_1)\un_{\ens{T_G\circ \theta_1<\infty}}$$\par
Ce qui veut simplement dire que le point de sortie de la trajectoire est le même une fois qu'on a ``gommé''
le premier point qui était dans $F$ et donc ne jouait pas de rôle. Et donc, si l'on pose 
$U(\omega)= g(X_{T_G(\omega)})\un_{\ens{T_G(\omega)<+\infty}}$, on a $\mathbb{P}_x$ p.s, $U= U\circ\theta_1$.
En utilisant la propriété de Markov simple, on a donc, pour $x\in F$~:\par
$$h(x) = \eesp{x}{U} = \eesp{x}{U\circ\theta_1} = \eesp{x}{\eesp{X_1}{U}} = \eesp{x}{h(X_1)}=Qh(x)$$\par
Ce qui montre que $h$ est harmonique sur $F$. Comme sur $G$, $T_G=0$, il est clair qu'on a $h=g$ sur $G$.
On admet l'unicité de la solution, qui nécessite un recours aux martingales, et allonge considérablement la 
preuve.\par
On peut alors poser clairement le problème de Dirichlet discret et le résoudre avec cette méthode.
On choisit $F$ une partie finie de ${\zz}^2$ (qui est bien dénombrable). Une fonction $h$ définie sur 
$\bar{F}$ est dite harmonique sur $F$ si pour tout $x$ de $F$, $h(x)$ est égal à la moyenne de $h$ sur les 
$4$ plus proches voisins de $x$. On retrouve la notion précédente en prenant comme chaîne de Markov la marche
aléatoire simple sur ${\zz}^2$~: $Q(x,x\pm e_i) = 1/4$. Alors en utilisant le théorème précédent, on obtient
que pour toute fonction $g$ positive définie sur $\partial F$, la seule fonction $h~:\bar{F}\to \rr_+$ qui 
soit harmonique sur $F$ et coincide avec $g$ sur $\partial F$ est donnée par~:\par
$$h(x) = \eesp{x}{g(X_{T_{\partial F}})}$$\par
o\`u\par
$$T_{\partial F} = \inf\ens{n\geq 0 : X_n\in \partial F}$$\par
En effet, d'après la fin de la première partie, le temps d'arrêt $T_{\partial F}$ est fini 
$\mathbb{P}_x$ presque surement puisque la marche aléatoire issue de $x$ visite tous les points du plan. 
(Une translation montre que cela reste vrai pour tout $x$).\par



\partie{Problème classique et traitement informatique.} 
\spartie{Mouvement Brownien.}
Tout d'abord, introduisons le mouvement brownien, qui sera le pendant
des marches aléatoires discrètes dans le cas continu.
\definition{} On appelle mouvement Brownien (en dimension 1, issu de 0), une famille
$(B_t)_{t\in \rr_+}$ de v.a. à valeurs réelles, vérifiant les deux propriétés suivantes~:\par
\begin{itemize}
\item[(1)]{} $B_0=0$ p.s. et pour tout $p\geq 0$ et tous réels $0=t_0<t_1<\ldots < t_p$, les 
v.a. $B_{t_1}, B_{t_2}-B_{t_1},\ldots, B_{t_p}-B_{t_{p-1}}$ sont indépendantes et pour tout 
$j$, $B_{t_j}-B_{t_{j-1}}$ a une loi gaussienne $\nnn(0,t_j-t_{j-1})$.
\item[(2)]{} Pour tout $\omega$, $t\to B_t(\omega)$ est continue.
\end{itemize}\par
\rmq Un mouvement Brownien en dimension $d$ est un vecteur $(B_1,\ldots,B_d)$ de mouvements
Browniens unidimensionnels.
\rmq Un mouvement Brownien issu de $x$ est $B+x$, o\`u $B$ est de la forme ci-dessus.
En utilisant la propriété que nous avons démontrée sur la convergence en loi de marches aléatoires, il apparaît qu'intuitivement
(et c'est le cas rigoureusement!), le mouvement Brownien est une ``limite'' de marches aléatoires convenablement changées d'échelle.
Cela explique la ressemblance des trajectoires obtenues pour une marche aléatoire et un brownien simulé~:\par
\centerline{\begin{tabular}{cc}
\includegraphics*[width=3cm,height=3cm]{images/marche_alea.ps} & 
\includegraphics*[width=3.5cm,height=3cm]{images/brownien_simul.ps}\\
Marche aléatoire & Mouvement Brownien simulé
\end{tabular}}\par
Notons que Caml ne génère pas de variable aléatoire gaussienne. Pour les obtenir, nous avons dû utiliser que si $U$ et $V$ sont
des variables aléatoires de loi uniforme sur $\segment{0}{1}$, alors $\sqrt{-2\ln(U)}\cos(2\pi V)$ a une loi gaussienne.\par
Pour ce qui est de la résolution du problème de Dirichlet classique avec des mouvements browniens, nous n'avons malheureusement pas
eu le temps (ni la capacité?) d'assimiler à temps les notions nécessaires. Nous admettons alors que la formule proposée
dans le cas discret continue de fonctionner, à condition de remplacer
$X_n$ par $B_t$ et $\mathbb{P}_x$ par une mesure de probabilité sur
${\ccc}({\rr}_+,{\rr}^d)$ qui est la mesure de Wiener.
Pour cela, il faut cependant que l'ouvert vérifie la condition de ``cône extérieur''. Cela signifie qu'en tout point de la frontière 
de l'ouvert, on peut placer un cône ne rencontrant pas l'ouvert.\par
\centerline{\includegraphics*[width=9cm,height=4.5cm]{images/Ouverts.eps}}
\spartie{Résolution informatique.}
\sspartie{Présentation de l'algorithme.}
Pour calculer l'espérance, qui donne la valeur de la solution au point
considéré, nous lançons $n$ marches aléatoires du point $x$, nous notons la valeur de la condition aux limites au point de sortie, puis nous faisons la moyenne de ces $n$ valeurs obtenues. C'est ce que réalise ce programme dans le cas discret~:\par
{\begin{verbatim}
#open "graphx";;
load_object "graphx";;
\end{verbatim}}\par
\uline{Calcul d'une valeur approchée de $\mathbb{E}_{(x,y)}(f(B_T))$~:}\par
{\begin{verbatim}
let dirichlet ouvert f x y n =
  let m = ref 0. in
    for i = 1 to n do  (* lancer de n marches aléatoires *)
      let a = ref x and b = ref y in
	while ouvert !a !b do
	  let u = random__int 4 in
	    if u = 0 then a := !a + 1
	    else if u = 1 then a := !a - 1
	    else if u = 2 then b := !b + 1
	    else b := !b - 1
	done;
	m := !m +. (f !a !b ); (* récupération de la valeur 
                                  de la condition aux limites
                                  au point de sortie *)
    done;
    !m /. (float_of_int n);;   (* renvoi du résultat *)
\end{verbatim}}\par
Ainsi, ``ouvert'' est une fonction de type \begin{verbatim}int-> int-> bool\end{verbatim} qui décrit
l'ouvert de définition, ``f'' est la condition aux limites et la fonction dirichlet est de type \begin{verbatim}(int-> int-> bool)-> (int->int->float)->int->int->int->float\end{verbatim}.\par
Voici une représentation de ce que fait la fonction dirichlet n fois~:\par
\centerline{\includegraphics*[width=5cm,height=5cm]{images/marche_sortie.ps}}
Afin d'obtenir une représentation graphique du résultat obtenu, nous stockons ces résultats dans une matrice par la fonction suivante~:\par
\begin{verbatim}
let solution ouvert f n =
  let t1 = borne_ouvert ouvert in (* fonction auxiliaire calculant
                               la taille de la matrice *)
    let m = mat_float (t1+1) in   (* fabrication de la matrice *)
    for x = 0 to t1 do
      for y = 0 to t1 do
	if ouvert x y then 
	  m.(x).(y) <- dirichlet ouvert f x y n (* remplissage *)
	else if frontiere ouvert x y then (* coloration en noir de
                                             la frontière *)
	  m.(x).(y) <- 0.     
	else m.(x).(y) <- 100. (* coloration en blanc du reste *)
      done;
    done;
    m;;
\end{verbatim}\par
o\`u~:\par
\begin{verbatim}
let frontiere ouvert x y =
  if not (ouvert x y) then
    if (ouvert (x+1) y)||(ouvert (x-1) y)||(ouvert x (y+1))||(ouvert x (y-1)) then true
    else false
  else false;;
\end{verbatim}\par
décrit la frontière de l'ouvert en question, en vue de la coloriser en noir, ce qui améliore la lisibilité.\par
On peut alors représenter la solution, en prenant une convention de coloration~:\par
\begin{verbatim}
let couleur i =
  match i with
    |0 -> black |1 -> red |2 -> yellow |3 -> green
    |4 -> blue |5 -> cyan |6 -> white;;
\end{verbatim}
Ces couleurs sont choisies pour représenter au mieux le cas o\`u le potentiel est la température. Le chaud étant alors représenté en blanc et le froid en noir.
On est alors en mesure d'écrire le programme suivant~:\par
\begin{verbatim}
let graphe_solution ouvert f n =
  let graphe_mat m =
  clear_graph();
  let t1 = vect_length m.(0) in
  let p = max m in (* détermination du max de la matrice *)
    for i = 0 to (t1 - 1) do
      for j = 0 to (t1 - 1) do
	let c = int_of_float (7. *. (m.(i).(j) /. (p +. 1.))) in
	  set_color (couleur c); (* affectation d'une couleur *)
	  fill_rect {x=i; y=j} 1 1; (* représentation *)
      done;
    done;
  in
  open_graph " 640x640";
  let m = solution  ouvert f n in
    graphe_mat m;;
\end{verbatim}\par
Donnons des exemples de résolution dans divers cas~:\par\vspace{2ex}
\begin{tabular}{ccc}
\includegraphics*[width=3cm,height=3cm]{images/drapeau.ps} & 
\includegraphics*[width=3.5cm,height=3cm]{images/triangle.ps} & 
\includegraphics*[width=3cm,height=3cm]{images/rond.ps}\\
\begin{minipage}{5cm}
Le potentiel au bord ne dépend que de $x$. On trouve bien un potentiel intérieur du même genre.
\end{minipage} & 
\begin{minipage}{5cm}
On se place sur un triangle dont un côté seulement est chauffé.
\end{minipage} &
\begin{minipage}{5cm}
On se place sur un disque dont on a chauffé un secteur angulaire.
\end{minipage}
\end{tabular}\par
Dans le cas continu, les programmes sont identiques à condition de remplacer la marche aléatoire par un brownien (il faut tout de même faire
quelques adaptations!).
\spartie{Complexité.}
Afin d'évaluer la complexité moyenne de cet algorithme, il convient de simplifier les choses. On introduit donc
$T_n = \inf\lbrace k : \va{S_k} = n\rbrace$ qui représente le temps de sortie de l'intervalle 
$\segment{-n}{n}$ d'une marche aléatoire de loi de variance $\sigma^2$. Calculons son 
espérance en montrant que $S_n^2-\sigma^2 n$ est une martingale pour la filtration naturelle du processus $X_n$ 
(i-e $\fff_n = \sigma(X_1, \ldots X_n)$). Ces variables sont clairement intégrables, et 
$\espc{S_{n+1}^2-(n+1)\sigma^2}{\fff_n} = \espc{X_{n+1}^2 +2X_{n+1}S_n+S_n^2 - (n+1)\sigma^2}{\fff_n}$. 
On utilise alors les propriétés de l'espérance conditionnelle et le fait que $S_{n+1}=S_n+X_{n+1}$~:\par
\begin{itemize}
\item{}$\espc{X_{n+1}^2}{\fff_n} = \esp{X_{n+1}^2} = \sigma^2$ par indépendance de $X_{n+1}$ par rapport à 
$\fff_n$.
\item{}$\espc{2X_{n+1}S_n}{\fff_n} = 2S_n\esp{X_{n+1}} = 0$ car $S_n$ est $\fff_n$ mesurable.
\item{}$\espc{S_n^2}{\fff_n} = S_n^2$ car $S_n$ est $\fff_n$ mesurable.
\end{itemize}
On obtient bien le résultat annnoncé. En appliquant le théorème d'arrêt (admis\ldots) à cette martingale, on montre que
le processus $S_{k\land T_n}^2 - \frac{1}{2} k\land T_n$ est une martingale. En prenant l'espérance, il vient~:
$\esp{S_{k\land T_n}^2}=\frac{1}{2}\esp{k\land T_n}$. Et une application du théorème de convergence dominée de 
Lebesgue à chacun des deux membres montre que~:\par
$$\esp{T_n} = \left(\dfrac{n}{\sigma}\right)^2$$\par
Cela montre que pour chaque point dont on calcule le potentiel approché, on réalise de l'ordre de $T^2$ 
opérations, o`u $T$ désigne la taille de l'ouvert. Comme on fait cela pour $T^2$ points à peu près, le programme
a une complexité en $n \mathcal{O}(T^4)$ o~`u $n$ désigne le nombre de calcul pour chaque point. C'est une 
complexité très mauvaise. Cependant, elle a l'avantage de ne pas dépendre de la géométrie de la frontière.






%\appendix
\section{Programmes en Caml}
\uline{Résolution du problème de Dirichlet discret en caml et représentation graphique des solutions.}\par
On se place en dimension $2$. Les programmes qui suivent fournissent une résolution numérique approchée du problème
discret tel qu'il est traité dans la partie $2$. Dans un souci de concision, nous ne faisons pas figurer des programmes
élémentaires comme la fabrication d'une matrice, la détermination de son maximum ou autres tâches qui sont
en fait de simples boucles for.\par
Le caml-toplevel à lancer est \textit{camllight camlgraph}.
Il faut commencer par appeler les bibliothèques graphiques de J.M. Quercia ce qu'on fait avec les commandes~:\par
{\begin{verbatim}
#open "graphx";;
load_object "graphx";;
\end{verbatim}}
\uline{Calcul d'une valeur approchée de $\mathbb{E}_{(x,y)}(f(B_T))$~:}\par
{\begin{verbatim}
let dirichlet ouvert f x y n =
  let m = ref 0. in
    for i = 1 to n do
      let a = ref x and b = ref y in
	while ouvert !a !b do
	  let u = random__int 4 in
	    if u = 0 then a := !a + 1
	    else if u = 1 then a := !a - 1
	    else if u = 2 then b := !b + 1
	    else b := !b - 1
	done;
	m := !m +. (f !a !b );
    done;
    !m /. (float_of_int n);;
\end{verbatim}}
\uline{Calcul de la taille de l'ouvert~:}\par
{\begin{verbatim}
let borne_ouvert ouvert =
  let t1 = ref 0 in
    for i = 0 to 639 do
      for j = 0 to 639 do
        if (ouvert i j) && (i > !t1) then t1 := i
        else if (ouvert i j) && (j > !t1) then t1 := j
      done;
    done;
    !t1 + 1;;
\end{verbatim}}

\uline{Frontière de l'ouvert~:}\par
{\begin{verbatim}
let frontiere ouvert x y =
  if not (ouvert x y) then
    if (ouvert (x+1) y)||(ouvert (x-1) y)||(ouvert x (y+1))||(ouvert x (y-1)) then true
    else false
  else false;;
\end{verbatim}}

\uline{Calcul de la matrice des températures approchées~:}
{\begin{verbatim}
let solution ouvert f n =
  let t1 = borne_ouvert ouvert in
    let m = mat_float (t1+1) in(*fabrication d'une matrice de flottants*)
    for x = 0 to t1 do
      for y = 0 to t1 do
	if ouvert x y then
	  m.(x).(y) <- dirichlet ouvert f x y n
	else if frontiere ouvert x y then
	  m.(x).(y) <- 0.
	else m.(x).(y) <- 100.
      done;
    done;
    m;;
\end{verbatim}}

\uline{Affectation des couleurs~:}\par
{\begin{verbatim}
let couleur i =
  match i with
    |0 -> black |1 -> red  |2 -> yellow |3 -> green
    |4 -> blue  |5 -> cyan |6 -> white;;
\end{verbatim}}

\uline{Représentation graphique d'une matrice de flottants~:}\par
{\begin{verbatim}
let graphe_mat m =
  clear_graph();
  let t1 = vect_length m.(0) in
  let u = max m in(*détermination du maximum de m*)
    for i = 0 to (t1 - 1) do
      for j = 0 to (t1 - 1) do
	let c = int_of_float (7. *. (m.(i).(j) /. (u +. 1.))) in
	  set_color (couleur c);
	  fill_rect {x=i; y=j} 1 1;
      done;
    done;;
\end{verbatim}}

\uline{Le programme terminal~:}\par
{\begin{verbatim}
let graphe_solution ouvert f n =
  open_graph " 640x640";
  let m = solution  ouvert f n in
    graphe_mat m;;
\end{verbatim}}
{\begin{verbatim}
graphe_solution :
 (int -> int -> bool) -> (int -> int -> float) -> int -> unit = <fun>
\end{verbatim}}



\begin{thebibliography}{a}
\bibitem{integration}
Walter \textsc{Rudin}.\emph{Analyse réelle et complexe}.
Dunod 1998

\bibitem{internet}
Internautes.\ \emph{http://les-mathématiques.net/forum}

\bibitem{proba}
J-F. \textsc{Le Gall}.\ \emph{Intégration, probabilités et processus aléatoires}.

\bibitem{Contact}
S. \textsc{Cohen}.
Chercheur au CICT, Université \textsc{Paul Sabatier}.


\end{thebibliography}
\end{document}












